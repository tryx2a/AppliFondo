\section{Numerical integration}
\subsection{Overview}

To use these functionalities, you should include \verb!pnl/pnl_integration.h!.

Numerical integration methods are designed to numerically evaluate the integral
over a finite or non finite interval (resp. over a square) of real valued
functions defined on $\R$ (resp. on $\R^2$).

\begin{verbatim}
typedef struct {
  double (*function) (double x, void *params);
  void *params;
} PnlFunc;

typedef struct {
  double (*function) (double x, double y, void *params);
  void *params;
} PnlFunc2D;
\end{verbatim}

We provide the following two macros to evaluate a \refstruct{PnlFunc} or
\refstruct{PnlFunc2D} at a given point
\begin{verbatim}
#define PNL_EVAL_FUNC(F, x) (*((F)->function))(x, (F)->params)
#define PNL_EVAL_FUNC2D(F, x, y) (*((F)->function))(x, y, (F)->params)
\end{verbatim}



\subsection{Functions}

\begin{itemize}
\item \describefun{double}{pnl_integration}{\refstruct{PnlFunc} \ptr F,
    double x0, double x1, int n, char \ptr meth}
  \sshortdescribe Evaluate $\int_{x_0}^{x_1} F$ using \var{n} discretization
  steps. The method used to discretize the integral is defined by \var{meth}
  which can be \var{"rect"} (rectangle rule), \var{"trap"} (trapezoidal rule),
  \var{"simpson"} (Simpson's rule).

\item \describefun{double}{pnl_integration_2d}{\refstruct{PnlFunc2D} \ptr F,
    double x0, double x1, double y0, double y1, int nx, int ny, char \ptr meth}
  \sshortdescribe Evaluate $\int_{[x_0, x_1] \times [y_0, y_1]} F$ using
  \var{nx} (resp. \var{ny}) discretization steps for \var{[x0, x1]}
  (resp. \var{[y0, y1]}). The method used to discretize the integral is
  defined by \var{meth} which can be \var{"rect"} (rectangle rule),
  \var{"trap"} (trapezoidal rule),   \var{"simpson"} (Simpson's rule).


\item \describefun{int}{pnl_integration_qng}{\refstruct{PnlFunc} \ptr F,
    double x0, double x1, double epsabs, double epsrel, double \ptr result,
    double \ptr abserr,  int \ptr neval}
  \sshortdescribe Evaluate $\int_{x_0}^{x_1} F$ with an absolute error less
  than \var{espabs} and a relative error less than \var{esprel}. The value of
  the integral is stored in \var{result}, while the variables \var{abserr} and
  \var{neval} respectively contain the absolute error and the number of function
  evaluations. This function returns \var{OK} if the required accuracy has been
  reached and \var{FAIL} otherwise. This function uses a non-adaptive Gauss
  Konrod procedure (qng routine from {\it QuadPack}).

\item \describefun{int}{pnl_integration_GK}{\refstruct{PnlFunc} \ptr F,
    double x0, double x1, double epsabs, double epsrel, double \ptr result,
    double \ptr abserr,  int \ptr neval}
  \sshortdescribe This function is a synonymous of
  \reffun{pnl_integration_qng} and is only available for backward
  compatibility. It is deprecated, please use \reffun{pnl_integration_qng}
  instead.

\item \describefun{int}{pnl_integration_qng_2d}{\refstruct{PnlFunc2D} \ptr F,
    double x0, double x1, double y0, double y1, double epsabs, double epsrel,
    double \ptr result, double \ptr abserr, int \ptr neval}
  \sshortdescribe Evaluate $\int_{[x_0, x_1] \times [y_0, y_1]} F$ with an
  absolute error less than \var{espabs} and a relative error less than
  \var{esprel}. The value of the integral is stored in \var{result}, while the
  variables \var{abserr} and \var{neval} respectively contain the absolute error
  and the number of function evaluations. This function returns \var{OK} if the
  required accuracy has been reached and \var{FAIL} otherwise.

\item \describefun{int}{pnl_integration_GK2D}{\refstruct{PnlFunc} \ptr F,
    double x0, double x1, double epsabs, double epsrel, double \ptr result,
    double \ptr abserr,  int \ptr neval}
  \sshortdescribe This function is a synonymous of
  \reffun{pnl_integration_qng_2d} and is only available for backward
  compatibility. It is deprecated, please use \reffun{pnl_integration_qng_2d}
  instead.

\item \describefun{int}{pnl_integration_qag}{\refstruct{PnlFunc} \ptr F,
    double x0, double x1, double epsabs, int limit, double epsrel, double \ptr
    result, double \ptr abserr,  int \ptr neval}
  \sshortdescribe Evaluate $\int_{x_0}^{x_1} F$ with an absolute error less
  than \var{espabs} and a relative error less than \var{esprel}. \var{x0} and
  \var{x1} can be non finite (i.e. \var{PNL_NEGINF} or \var{PNL_POSINF}). The
  value of the integral is stored in \var{result}, while the variables
  \var{abserr} and \var{neval} respectively contain the absolute error and the
  number of iterations. \var{limit} is the maximum number of subdivisions of the
  interval \var{(x0,x1)} used during the integration. If on input, \var{limit
    0}, then 750 subdivisions are used.  This function returns \var{OK} if the
  required accuracy has been reached and \var{FAIL} otherwise. This function
  uses some adaptive procedures (qags and qagi routines from {\it QuadPack}).
  This function is able to handle functions \var{F} with integrable
  singularities on the interval \var{[x0,x1]}.

\item \describefun{int}{pnl_integration_qagp}{\refstruct{PnlFunc} \ptr F,
    double x0, double x1, const PnlVect \ptr singularities, double epsabs,
    int limit, double epsrel, double \ptr result, double \ptr abserr,  int \ptr neval}
  \sshortdescribe Evaluate $\int_{x_0}^{x_1} F$  for a function
  \var{F} with known singularities listed in \var{singularities}.
  \var{singularities} must be a sorted vector which does not contain \var{x0}
  nor \var{x1}.  \var{x0} and \var{x1} must be  finite. The value of the
  integral is stored in \var{result}, while the variables \var{abserr} and
  \var{neval} respectively contain the absolute error and the number of
  iterations. \var{limit} is the maximum number of subdivisions of the interval
  \var{(x0,x1)} used during the integration. If on input, \var{limit = 0}, then
  750 subdivisions are used.  This function returns \var{OK} if the required
  accuracy has been reached and \var{FAIL} otherwise. This function uses some
  adaptive procedures (qagp routine from {\it QuadPack}).  This function is
  able to handle functions \var{F} with integrable singularities on the interval
  \var{[x0,x1]}.
\end{itemize}

% vim:spelllang=en:spell:

